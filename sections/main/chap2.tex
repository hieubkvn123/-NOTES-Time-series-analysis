\newpage
\section{Stationary time-series}
\subsection{Strict vs. Weak stationarity}

\begin{definition}[Mean, covariance and variance functions]
    Given a time-series $(X_t)_{t\in T}$ (where $T$ is a time domain), we have the following functions:
    \begin{itemize}
        \item Mean : $\mu(t)=\mathbb{E}[X_t], \ t\in T$.
        \item Covariance : $\gamma(t, s) = Cov(X_t, X_s), \ t, s\in T$.
        \item Variance : $\sigma^2(t) = \gamma(t,t)= Var(X_t), \ t\in T$.
    \end{itemize}
\end{definition}


\begin{definition}[Strict stationarity]
    A time-series is said to be \textbf{strictly stationary} if, $\forall n \in \mathbb{Z}_+, h \in \mathbb{N}$, we have:
    \begin{align*}
        (X_1, \dots, X_n) \stackrel{d}{\simeq} (X_{1+h}, \dots, X_{n+h}) 
    \end{align*}

    \noindent Meaning the two tuples are equal in distribution.
\end{definition}

\noindent \textbf{Remark} : Note that in practice, strict stationarity is often hard to achieve. Hence, we have some relaxed conditions for stationarity.

\begin{definition}[Weak stationarity]
    A time-series $(X_t)_{t\ge0}$ is said to be \textbf{weakly stationary} if the following two conditions are met:
    \begin{itemize}
        \item $\mu(t)$ does not depend on $t$.
        \item $\gamma(t, h)$ does not depend on $t$ for all $h\in\mathbb{N}$.
    \end{itemize}
\end{definition}

\subsection{Moving Average - $\bf MA(q)$ model}
\begin{definition}[White noise]
    A process $(Z_t)_{t\in T}$ is called \textbf{white noise} if it satisfies:
    \begin{itemize}
        \item $Cov(Z_t, Z_s)$ for all $s\ne t$.
        \item $Var(Z_t) = \sigma^2, \ \forall t\in T$.
    \end{itemize}

    \noindent We denote that $Z_t\sim WN(0, \sigma^2)$.
\end{definition}

\begin{definition}[$\bf MA(q)$ model]
    
\end{definition}
